\documentclass[draft]{article}

\usepackage{csquotes}
\usepackage[colorinlistoftodos,prependcaption,textsize=tiny,disable]{todonotes}
\usepackage{palatino}
\usepackage{graphicx}
\usepackage{float}
\usepackage{enumitem}
\usepackage{cleveref}
\usepackage[style=authoryear,natbib=true,backend=bibtex]{biblatex}

\setcounter{secnumdepth}{3}
\crefname{chapter}{\S}{\S\S}
\crefname{section}{\S}{\S\S}
\crefname{table}{table}{tables}
\Crefname{table}{Table}{Tables}
\crefname{figure}{figure}{figures}
\Crefname{figure}{Figure}{Figures}
\crefname{appendix}{appendix}{appendices}
\Crefname{appendix}{Appendix}{Appendices}

\bibliography{biblio}


\begin{document}

%%%%%%%%%%%%%%%%%%%%%%%%%%%%%%%%%%%%%%%%%%%%%%%%%%%%%%%%%%%%%%%%%%%%%%%%%%%%%%%%
\author{Tom Wallis}
\title{On Anthropomorphic Algorithms}  % Maybe "On Anthropomorphic Algorithms as they Pertain to the Space of Possible Minds"?
\date{}
\maketitle
%%%%%%%%%%%%%%%%%%%%%%%%%%%%%%%%%%%%%%%%%%%%%%%%%%%%%%%%%%%%%%%%%%%%%%%%%%%%%%%%

\tableofcontents
\newpage

\begin{abstract} %%  Perhaps this should be rewritten once the essay draft is complete...
    In philosophy of mind, a recurring topic of philosophical research and popular philosophy is that of the ``mind'' an artificially intelligent agent might possess. A popular method for categorising typical artificially intelligent agents is John Searle's ``weak'' versus ``strong'' AI, where he differentiates between acting intelligently (a ``weak'' AI) and having a mind and mental states (a ``strong'' AI). In this essay, an alternative method for approaching the hard problem of consciousness is presented. This is arrived at by augmenting Sloman's concept of the space of possible minds~\citep{Sloman1984TheMinds} through arguments made using simple yet interesting recent computing science techniques. This argument is explored by applying it as a potential solution to concrete problems of AI safety, such as the problem of Corrigibility~\citep{corrigibility} and Reward Hacking~\citep{concrete_problems}. The efficacy and practical application of the technique is also assessed.
\end{abstract}


%%%%%%%%%%%%%%%%%%%%%%%%%%%%%%%%%%%%%%%%%%%%%%%%%%%%%%%%%%%%%%%%%%%%%%%%%%%%%%%%
\section{Problem Outline}
\subsection{Existential Risk and AI Safety}
Research on existential risk has increasingly turned an eye toward problems of safety regarding artificial intelligence. This research suffers some difficult challenges. For one, practical exploration of what is often termed ``strong AI'' --- an artificially intelligent agent which has a ``mind'' a mental states --- cannot be explored by concrete example. Rather, researchers must obliquely attack the problem by observing how minds in humans (and other conscious animals) appear to operate.\par

Some examples of ways to approach this problem present useful tools to the curious philosopher.~\cite{Sloman1984TheMinds} presents an approach whereby a space of possible minds is envisaged. This approach is useful when describing some of AI safety's most interesting problems, such as the paperclip maximiser~\citep{bostrom2003ethical}. The principle of the argument is that, when giving a superintelligent agent the task of making as many paperclips as it can, it will consume any and all resources present for the construction of paperclips --- including, for example, iron in human blood.\par

Whether the paperclip maximiser problem is likely to occur is irrelevant; it shows that there are issues with the behaviour we might expect a very intelligent agent to express. More recently, concrete issues in artificial intelligence safety have been identified which researchers can work on today. \par

An example is found in reward hacking~\citep{concrete_problems}, where an agent's behaviour might tend toward unintended end-states so as to satisfy the criteria for its fulfilment of a goal, without achieving the goal intended to be set for the agent. An example might be an agent told to tidy mess left by a child, which identifies a messy room by whether it can see objects lying on the floor. This agent may ``achieve'' its goals by detaching its robotic eyes. As it cannot see any objects lying on the floor, it must conclude that its goal has been achieved --- but mess can still be found on the floor. This particular behaviour seems harmless, but unintended behaviour and end-states might end badly for human agents in other situations. Another example might be that of agent corrigibility~\citep{corrigibility}, where an agent acting in unexpected ways might not accept corrective intervention on the part of its creators. This agent might note that a change to its goals does not help to achieve its currently assigned goal; therefore, to achieve its goal, it is inclined to ignore attempts to alter its behaviour by its creators.\par

\subsection{A Philosophical Approach to Attacking the Problems}
Existential Risk issues are often approached from a philosophical angle, shedding light on important issues and directing thought on the issue toward likely solutions. In AI safety, philosophical approaches are of paramount importance, as sufficiently advances intelligent systems have not been developed to such a degree that their intellect might be comparable to an ordinary human mind. The empirical assessment of general intelligence might come about as techniques such as whole brain emulation become viable for humans, or as general artificial intelligence comes to the fore as computer-scientific techniques advance.\par

Tools like Sloman's Space of Possible Minds\footnote{For brevity and readability, Sloman's Space of Possible Minds will be referred to as simply ``Sloman's space'' through this essay.} can be of some help in reasoning about artificial intelligence. Though Sloman doesn't state much about practical structures of his space, thinking in terms of the ``dimensions'' a mind might have --- its social capability, or its sense of self preservation, for example --- help to reason about the differences different mind types might have. A machine might not have much reason to preserve itself, particularly when considering that what makes it conscious can presumably be backed up, or that multiple copies of it attempting to accomplish identical goals exist\footnote{It's worth noting that some theorise that sufficiently intelligent minds converge on similar properties, such as self preservation. ``Instrumental Convergence'' as it pertains to artificial intelligence is well explored in~\cite{basic_ai_drives}.}. Similarities and differences between minds would be geometric differences within the space; therefore, minds with similar qualities would appear closer together than others in the theoretical space, and the closer two minds were, the closer their qualities would be\footnote{The space's natural geometric properties are a useful feature of Sloman's approach. For example, one might be inclined to take the euclidean distance between two minds in the space as a naive measure of how different they are.}.\par

Unfortunately, reliably determining the structure and nature of Sloman's space, with little empirical work possible, and with the very nature of a ``mind'' a philosophical quandary, is an impossible feat today. Therefore, as a technique, Sloman's space has issues which limit how practically a philosopher might reason using it.\par

\subsection{Introducing Anthropomorphic Algorithms}
Anthropomorphic Algorithms are algorithms designed to guide an intelligent agent's behaviour in more human-like ways. Existing Anthropomorphic Algorithms include Marsh's formalism of trust~\citep{Marsh1994FormalisingConcept} or Eigentrust~\citep{eigentrust}, which both simulate ``trusting'' behaviour. Indeed, trust formalisms are possibly the most widely researched anthropomorphic algorithms. Some formalisms of trust --- such as Marsh's --- attempt to describe the sociological and psychological factors of trust in humans, and later use this formalism to quantify trust in some way, building algorithms around the quantification such that intelligent agents might exhibit the same behaviour\footnote{As anthropomorphic algorithms are generally implementations of a formalism of some human behavioural trait, ``anthropomorphic algorithm'' and ``formalism'' will often be used interchangeably through this document.}. This means that, while their applications in computing science are broad, they have applications in other fields also --- such as the social sciences.\par

The anthropomorphic algorithm's very existence, and the formalism of a certain behaviour across different types of minds, presents an opportunity for philosophers to make use of Sloman's space as a powerful tool in reasoning about AI safety, and about theories of mind generally. \todo{SPECIFY WHAT I'LL DETAIL IN EACH SECTION AND WHAT WILL BE PRESENTED THROUGH THE ESSAY}

\section{Assertions}

\section{Applying Anthropomorphic Algorithms to Sloman's Space}

Sloman's space can be envisaged as a literal mathematical space, where every point describes a certain mind. We can see that there is no guarantee that a subspace of human minds and a subspace of artificial minds --- minds born from simple intelligent agents, for example --- will have anything in common mathematically. Certainly, one cannot assert that numerically identical minds exist in both subspaces, as this would require them to overlap: two numerically identical minds would, by definition, occupy the same point in Sloman's space. As we can assert nothing about the structure of Sloman's space, this overlap has no guarantee.\par

However, Sloman's space describes the fundamental properties of a mind; emergent phenomena are unsuitable as dimensions of the space, as they can be affected by many independent variables. Note also that these emergent phenomena can arise in the same way in different minds. One might say, ``Alice and Bob are just as greedy'', without implying anything about fundamental traits of Alice and Bob. ``Greed'' is a property they share, but their greed can arise for a plethora of reasons --- because greed is a phenomenon which emerges from fundamental traits of their minds.\par

This is true of other traits, too. For example, according to Castelfranchi \& Falcone~\citep{CastelfranchiSocialApproach}, trust arises from an assessment of the capability and will of a trusted agent to achieve goals of the trustee. Luhmann, a sociologist, asserts that trust is a form of social risk aversion~\citep{luhmann2000familiarity}, while Deutsch, a psychologist, indicates that trust incorporates utility as well as risk in different ways, expressed as confidence, gambling, masochism and other behavioural tendencies~\citep{deutsch1962cooperation}.\par

Trust is a well-studied topic with a range of literature to be drawn from. More importantly to the argument at hand, it is an emergent phenomena which manifests in different ways and is born from no clear underlying properties of a mind. However, formalisms of trust --- and how trusting agents behave --- can be found in Castelfranchi and Falcone's work, as well as Marsh's, and plenty of others. One can assert that these formalisms describe an aspect of behaviour in humans, with the exception of some types of minds such as human sociopaths, whose behaviour one might expect to be abnormal: this is precisely their goal. Every formalism listed, however, is also an algorithmic formalism, meaning that it can be implemented so as to direct the behaviour of an intelligent agent in a more anthropomorphic way. They are therefore the canonical example of the anthropomorphic algorithm, and happen to be the most extensively studied.\par

A given anthropomorphic algorithm wouldn't apply to any type of mind: caveats such as sociopaths are to be expected, and so a domain --- a subspace of Sloman's space --- is suitable as a way to limit the scope of the formalism. Trust formalisms could then be seen as algorithms (or ordinary mathematical functions) over a subspace of minds which describe behaviour. However, from existing anthropomorphic algorithms such as trust, we can see that this domain hints toward making Sloman's space more useful philosophically: the domain of a trust formalism such as Marsh's is to apply to not only neuronormative humans, but intelligent agents, also.\par

The mind of an intelligent agent and a human may be markedly different, but they \emph{can} give rise to the same emergent phenomena. Importantly, the formalisms might manifest in numerically non-identical ways: the way an intelligent agent computes trust is rather different to how the human brain does, even under the same formalism, due to differences in hardware and in implementation detail. However, they conform to the same description of the phenomenon they exhibit --- the formalism --- and should behave in the same way as a result. One might say that the minds trust in qualitatively identical ways, and follow a numerically identical formalism, with differences arising in implementation of the overarching theory.\par

Using anthropomorphic algorithms, then, the philosopher can assert behavioural identity between different types of minds. This abstraction over Sloman's space means that the space's indeterminate structure no longer prevents asserting these identities, precisely because the space --- regardless of structure --- allows for the definition of the domains of formalisms which describe an agent's behaviour.\par

\subsection{Compatibility with Existing Literature}  %%  Is there only one thing in theory of mind we relate this to?

\subsubsection{Physical Functionalism}  %% Should non-reductive physicalism be here or separate?

\section{Applications of the Theory}
The theory proposed can be applied in several ways, due to its interdisciplinary nature. Detailed below are some examples, focusing on the problem of AI Safety as an example of its utility in solving practical, concrete problems in a largely theoretical field.\par

\subsection{Attacking Concrete AI Safety Problems}

\subsubsection{Reward Hacking}
A list of concrete AI safety problems researchers could work on solving at present is described in~\cite{concrete_problems}. Among those is the problem of Reward Hacking. In~\cite{concrete_problems}, reward hacking is introduced as:

\begin{displayquote}
In ``reward hacking'', the objective function that the designer [of the AI] writes down admits of some clever ``easy'' solution that formally maximizes it but perverts the spirit of the designer’s intent (i.e. the objective function can be ``gamed''), a generalization of the wireheading problem.
\end{displayquote}

Reward hacking exists in humans. For example, some people are known to pirate software: in this instance, the objective function is to acquire software, and this is achieved through piracy. It also allows the human to acquire \emph{more} software, as piracy uses far less resources than paying for computer programs --- this is akin to the formal maximisation quoted. This circumvents the societally intended method for acquiring software, however, which is to pay for it. As a result, the developer of the software is left unsupported for the work put into creating the pirated software.\par

Often, humans will not pirate software and will instead pay. In some scenarios, this behaviour is made more likely through adapting the context of the goal: operating systems with easier-to-use stores for software, or tight measures for signature and checking of authenticity of a program to discourage piracy. Other humans are less inclined to pirate than other, however --- there are therefore behavioural tendencies which limit the inclination of a mind to reward hack. Examples of this might include social responsibility or a decreased degree of comfort in actions which harm other agents in some way. Happily, computational comfort formalisms are already being developed~\citep{Marsh2011}, and computational responsibility formalisms are currently being worked on~\cite{wallis_responsibility}.\par

\subsubsection{Corrigibility}
The problem of an agent's corrigibility is defined in~\cite{corrigibility} as:

\begin{displayquote}
We call an AI system “corrigible” if it cooperates with what its creators regard as a corrective intervention, despite default incentives for rational agents to resist attempts to shut them down or modify their preferences.
\end{displayquote}



\subsection{Anthropology}


\subsection{Human-Computer Interaction} % The machine isn't the only thing being formalised. Interaction from both angles can be described more formally, which is important given there's not much more than Fitt's law to HCI formalisms & mathematics. 

\section{Future Work}\label{sec:future_work}


\subsection{Decision Theory}  % How do we *combine* traits?


\subsection{Developing Computational Formalisms}  % We need more than just trust. step 1, identify new formalisms to make. Step 2, create and test new formalisms. 


\subsection{The Problem of Multiple Formalisms}  % We can specify many formalisms for one human trait. Which formalisms are more correct? How do we talk about the qualities of the behaviour the formalism describes, and how it differs from other formalism's descriptions?

\section{Discussion}

%%%%%%%%%%%%%%%%%%%%%%%%%%%%%%%%%%%%%%%%%%%%%%%%%%%%%%%%%%%%%%%%%%%%%%%%%%%%%%%%

\printbibliography

\end{document}
