\section{Discussion}
Behaviour can be modelled as an abstraction over Sloman's space, realised by formalisms of human behaviour which can also apply to non-human minds. This approach requires some development to be practically useful, but it can aid in providing a mitigating technique which makes an artificial intelligence's behaviour easier for a human developer to predict and reason about.\par

The technique has a number of benefits, including:

\begin{itemize}
  \item Helping to attack problems in AI safety, such as corrigibility and reward hacking
  \item Providing further opportunities for interdisciplinary study
  \item Opportunities to develop practical solutions to previously theoretical problems
  \item Progressing theories of mind into testable, implementable theories
\end{itemize}

The practical approach requires further development in both theory and application, including:

\begin{itemize}
  \item Development of more formalisms, from sociology, psychology, and anthropology
  \item Analysis of possible structures of Sloman's space
  \item Application of the formalisms to more AI safety problems
  \item Work on combining the behaviours described by formalisms of different behaviours
  \item Analysis of the theory as it pertains to theories of mind other than non-reductive physicalism
\end{itemize}

The theory itself, and the application of it as an appropriate model of behaviour, requires further argument and analysis. However, the theory integrates easily with existing and popular theories of mind, and popular concepts in AI safety literature (particularly Sloman's space).\par

What remains to be done, therefore, is not so much proof-of-concept work for the theory as it is the theory's applications. As these applications are presently realisable, a great body of work in a series of fields should be undertaken so as to test the hypothesis, and provide fertile ground for the development of further pragmatic approaches to AI safety and the integration of philosophy and computing science as a whole.\par
